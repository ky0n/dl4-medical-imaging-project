\documentclass{article}
\usepackage{listings}
\usepackage{hyperref}
\usepackage{graphicx}

\author{Tobias Dorra, Hendrik Schick}
\title{Project for Deep learning in medical imaging: Segmentation - MIC \\ \begin{large} 
Task 3: Implementation II
\end{large}}

\begin{document}
	
	\maketitle

	\section{Task}

		The task was to continue working on the Implementation.

	\section{Implementation}
On the last week we made various changes to the code.

We added a question at the beginning for the user to decide if the model should be trained with the liver data or the prostate data. 
We fixed the usage of the spacing and are using the spacing now correctly to rescale the images to a common size. This way the training is a lot more efficient and overall better.
\\
\newline We also configured a virtual machine on the Google cloud platform to train  the data with more RAM and a strong GPU. In order to train with GPU we had to install the tensorflow-gpu pip package and CUDA aswell as cuDNN for the graphic card. The training time was still several hours long with the current hyperparameters.
\\\\
The changes also include bugfixes for not matching shapes of the multidimensional arrays of the data used to train the model and also edge cases in for loops.
We also improved the overall logging and added logging for the training part to see the loss during the training.
The comments for the code have also been improved.
	\section{Problems and future tasks}
We still need to achieve higher accuracy with the model and want to tune hyperparameters. In order to achieve this we will also look for better architecture setup and possible mistakes.

We still want to train the model on the GPU machines and are trying it on friday, the 24.01.

\end{document}
